\chapter{Zerkleinern}
\begin{itemize}
	\item Älteste Verfahrenstechnik (prätechnologisch)
	
	\begin{itemize}
		\item Kauen von Nahrung
		\item Zerkleinern von Getreide im Mörser
	\end{itemize}
\end{itemize}

\section{Was ist "`Zerkleinern"'?}

\textbf{Prozessziel:}\\
Feststoff (aber auch Flüssigkeiten oder Gase) mit vertretbaren Energieaufwand (Betriebskosten) und erträglichen Verschleiß (Wartungskosten) auf eine gewünschte Feinheit (Dispersitätszustand) nach Produktspezifikationen zu bringen.\\
+ Anschaffungskosten\\

\textbf{Was kann zerkleinert werden ?}
\begin{enumerate}
	\item Getreide $\rightarrow$ Mehl, Gries, Flocken, Schrot, Spreu,...
	\item Gestein $\rightarrow$ Sand, Kies, Splitt, Zement,...
	\item Holz $\rightarrow$ Mulch, Spähne, Pallets, Spanplatten, OSB-Platten, Papier, Furnier,...
\end{enumerate}

\textbf{Wozu wird zerkleinert ?}
\begin{itemize}
	\item Erzeugen einer geünschten, bestimmten Korngrößenverteilung (evtl. mit $x_{min}$ und $x_{max}$)
	\item vergrößern der spezifischen Oberfläche $\left[ \si{\raiseto{2}\meter\per\raiseto{3}\meter} \right] \Rightarrow$ Reaktivität$\uparrow$
	\item Freilegen und Aufschließen einer Wertstoffphase (z.B. Erz)
	\item Struktur- und Formänderung (z.B. Haferflocken)
	\item mechanische Aktivierung
	\item Veränderung von Stoffeigenschaften nach Produktspezifikation:
		\begin{itemize}
			\item Fließverhalten, Transportfähigkeit, Dosierfähigkeit, Lagerfähigkeit
			\item Lösegeschwindigkeit, Reaktionsgeschwindigkeit, Extrahierfähigkeit
			\item Farbe, Oberfläche, Form, Raumfüllung
			\item ...
		\end{itemize}
\end{itemize}

\newpage

\section{Feststoff zerkleinern}
Einteilung erfolgt nach Größe des \underline{Produkts}:\\ \\
\vspace*{-5mm}
\renewcommand{\arraystretch}{1.2}
\begin{table}[h!]
	\centering
	\begin{tabulary}{\textwidth}{l|C}
		\textbf{Brechen:} & $5-50 \si{\milli\meter}$: fein \\ 
		&  $>50 \si{\milli\meter}$: grob\\ 
		\hline  
		& \\
		\textbf{Mahlen:} & $0,5-50 \si{\milli\meter}$: grob \\ 
		& $50 \si{\micro \meter}-500 \si{\micro \meter}$: fein \\ 
		& $5 \si{\micro \meter}-50 \si{\micro \meter}$: feinst \\ 
		& $<5 \si{\micro \meter}$: kolloid \\ 
	\end{tabulary} 
\end{table}
\FloatBarrier

\textbf{Ziel:} Überwinden der inneren Bindungskräfte $\rightarrow$ Bruch\\

\textbf{\large{mechanische Beanspruchung:}}
\begin{itemize}
	\item Druck
	\item Reibung
	\item Schlag
	\item Prall
	\item gegenseitiger Partikelstoß
	\item Schneiden (spalten)
	\item Scheren
	\item Scherströmung (für Tropfen, Mikroorganismen,..)
	\item Druckwelle (z.B. Sprengung)
	\item Kavitation (implodierende Dampfblase, bei der Teilchen herausgerissen wird)
\end{itemize}
\vspace*{5mm}
\textbf{\large{nicht-mechanische Beanspruchung:}}\\
d.h. Energiezufuhr
\begin{itemize}
	\item chemisch
	\item elektrisch
	\item thermisch
\end{itemize}

\newpage

\section{Energieaufwand von Mühlen (Zerkleinerungsmaschinen)}
\textbf{\large{Ziele:}}
\begin{itemize}
	\item Berechnung der Antriebsleistung einer Mühle ist abhängig von:
	\begin{itemize}
		\item Durchsatz
		\item Art des Stoffes
		\item Teilchenspezifikation (Korngröße)
	\end{itemize}
	\item Bauarten und Auswahl von Mühlen
\end{itemize}

\textbf{\large{spezifische Zerkleinerungsarbeit e:}}\\
\begin{equation}
	\text{e}=\si{\watt \per \meter} \left[\si{\joule \per \kilogram}\right]
\end{equation}\\
erweitern mit $\frac{1}{t}$
\begin{equation}
	\text{e}=\frac{\si{\watt\per t}}{\si{\meter \per t}}= \frac{P}{\dot{m}} \left[\si{\watt\per\kg\per \raiseto{-1}\second}\right]
\end{equation}

\textbf{Abhängigkeit von der Stoffeigenschaft:}\\
charakterisiert durch eine Materialkonstante
\begin{center}
	$c_B$ (Bondkonstante: experimentell bestimmt)
\end{center}

\textbf{Abhängigkeit von der Partikelgröße:}\\
charakteristische Teilchengröße
\begin{center}
	$X_{80} $ d.h. $H(x_{80})=80\%$ Durchgang
\end{center}

HIER STEHT IHR BILD\\ \\

$\rightarrow$ restliche 20\% werden meist ausgesiebt und wieder zurückgeführt "`80-20-Regel"'\\

\newpage

Die Modellierung von Zerkleinerungsprozessen ist äußerst komplex. Deshalb werden empirische Abschätzungsgleichungen verwendet ($\pm 50\%$ Genauigkeit).\\
Nur bei idealen Einzelkörnern kann man eine Bruchfunktion analytisch annähern.


\vspace*{-.55mm}
\renewcommand{\arraystretch}{1.2}

\begin{table}[h!]
		\centering
		\resizebox{0.95\textwidth}{!}{
		\begin{tabulary}{\textwidth}{C|C|C|C}
			\textbf{Name} & \textbf{Anwendung}&\textbf{Gleichung}&\textbf{Stoffkonstante} \\ 
			\hline  
			KICK& $x_{80_\omega}>\SI{50}{\milli\meter}$ &$e_{KICK}=c_K*log(\frac{x_{80_\alpha}}{x_{80_\omega}})$&$c_K=1,15*\frac{c_B}{\sqrt{0,05\si{\meter}}} \left[\si{\raiseto{2}\meter\per\raiseto{2}\second}\right]$\\
			BOND&$\SI{50}{\micro\meter}<x_{80_\omega}<\SI{50}{\milli\meter}$&$e_{BOND}=c_B*\left(\frac{1}{\sqrt{x_{80_\omega}}}-\frac{1}{\sqrt{x_{80_\alpha}}}\right)$& $c_B$: tabelliert $\left[\si{\raiseto{2,5}\meter\per\raiseto{2}\second}\right]$\\ 
			RITTER& $x_{80_\omega}>\SI{50}{\micro\meter}$&$e_{RITT}=c_R*\left(\frac{1}{x_{80_\omega}}-\frac{1}{x_{80_\alpha}}\right)$&$c_R= 0,5*c_B*\sqrt{\SI{5e-5}{\meter}}$ \\  
		\end{tabulary}}
\end{table}
\FloatBarrier

\textbf{Hinweise:}
\begin{itemize}
	\item $\alpha$: Anfangsgröße am Eingang
	\item $\omega$: Endgröße am Ausgang
	\item Teilchengröße \underline{\textbf{immer}} als $\left[\si{\meter}\right]$ einsetzen!
\end{itemize}

\textbf{Zerkleinerungsstrahl:}

%Start
\begin{figure}[h!]
	\centering
	\includegraphics[width=0.60\textwidth]{img/zerkleinerungsstrahl}
	\caption{Zerkleinerungsstrahl}
	\label{zerkleinerungsstrahl}
\end{figure}
\FloatBarrier
%Ende

$\mathbf{c_B}$\textbf{-Beispiele:}

\begin{table}[h!]
	\centering
		\begin{tabulary}{\textwidth}{C|C}
			\hline
			Kohle: & \SI{548}{\raiseto{2,5}\meter\per\raiseto{2}\second} \\   
			Gips:& \SI{394}{\raiseto{2,5}\meter\per\raiseto{2}\second} \\
			Eisenerz:&\SI{745}{\raiseto{2,5}\meter\per\raiseto{2}\second}\\ 
			gebr. Ton:&  \SI{69}{\raiseto{2,5}\meter\per\raiseto{2}\second}\\  
			Glimmer (Mineral):&\SI{6488}{\raiseto{2,5}\meter\per\raiseto{2}\second}\\
			\hline
	\end{tabulary}
\end{table}
\FloatBarrier

\underline{\textbf{meist:}} $c_B$ für trockenes Mahlen $> c_B$ für nasses Mahlen \\ \\

\large{\textbf{Beispielaufgabe: Zerkleinern}}

\newpage

\textbf{Energieaufwand beim Zerkleinern}
\begin{itemize}
 	\item Zerkleinern ist eine sehr energieintensive Grundoperation, deshalb hohe Betriebs- und Wartungskosten
 	\item ca. 5\% der Weltenergieerzeugung für Zerkleinerung
 	\item Zementherstellung sind 25\% der Kosten für Zerkleinerung
\end{itemize}
Energie ist nötig für:
\begin{itemize}
	\item Überwinden der inneren Bindungskräfte im Kern
	\item Reibung der Teilchen untereinander und im Apparat (Dissipation)
	\item kinet. Energie des Mahlprodukts
	\item Maschinenteil verschleißen
	\item Deformation der Teilchen ohne Bruch
	\item nicht ideale Einbringung der Kräfte (schiefer Stoß)	
\end{itemize}
$\Rightarrow$ Energieeffizienz der Zerkleinerung < 1\% 

%Tabelle START
\vspace*{-2.5mm}
\renewcommand{\arraystretch}{1.2}
\begin{table}[h!]
	\centering
	\caption{Vor- und Nachteile Trocken-/Nassmahlen}
	\begin{tabulary}{20cm}{C|C|C}
		\hline
		&\textbf{Trockenmahlen}  &\textbf{Nassmahlen} \\ 
		\hline
		&&\\
		\textbf{Vorteile}&\begin{minipage}[t]{0.4\textwidth}
			\begin{itemize}
				\item Gut ist trocken
			\end{itemize}
		\end{minipage} & 
		\begin{minipage}[t]{0.4\textwidth}
			\begin{itemize}
				\item geringerer Energiebedarf
				\item keine Staubentwicklung
				\item Kühlung des Produkts entgegen der Reibung 
			\end{itemize}
		\end{minipage}\\
	\hline
	&&\\
	\textbf{Nachteile} &\begin{minipage}[t]{0.4\textwidth}
		\begin{itemize}
			\item hoher Energiebedarf
			\item Staubentwicklung
			\item keine Kühlung des Produkts entgegen der Reibung
		\end{itemize}
	\end{minipage}&\begin{minipage}[t]{0.4\textwidth}
	\begin{itemize}
		\item Gut ist nicht trocken
	\end{itemize}
\end{minipage}\\
	\end{tabulary}
\end{table}
\FloatBarrier
\vspace*{-2.5mm}
%Tabelle ENDE

\section{Bauarten von Mühlen}
\begin{itemize}
	\item Backenbrecher
	\item Rundbrecher, Kegelbrecher
	\item \textbf{Kugelmühle}
	\begin{itemize}
		\item \textit{Kaskadenbewegung}\\
		Beanspruchung: Reibung $\rightarrow$ $n=0,6...0,7*n_{Krit}$
		\item \textit{Kateraktbewegung}\\
		Beanspruchung: Reibung \underline{und} Schlag $\rightarrow$ $n=0,8...0,9*n_{Krit}$
	\end{itemize}
	\textbf{Bestimmung der Grenzdrehzahl:}
	\begin{flalign}
		& F_G=F_Z\\
		&m*g=m*r*\omega^2 \text{ mit } \omega = 2*\pi*n \text{ (n... Drehzahl) }\\
		&g=r*4*\pi^2*n^2\\
		&n_{Krit}=\sqrt{\frac{g}{4*\pi^2*r}}\approx \sqrt{\frac{1\left[\si{\meter \per \raiseto{2}\second}\right]}{4*\pi^2*r}}=\frac{1\left[\sqrt{m}\right]}{\sqrt{2*D}}\\
	\end{flalign}
	\textbf{Vorsicht mit den Einheiten !} $n_{Krit}=\left[\frac{1}{s}\right]$ 
	%Tabelle START
	\vspace*{-2.5mm}
	\renewcommand{\arraystretch}{1.2}
	\begin{table}[h!]
		\centering
		\caption{Vor- und Nachteile der Kugelmühle}
		\resizebox{0.65\textwidth}{!}{
		\begin{tabulary}{\textwidth}{C|C}
			\hline
			\textbf{Vorteile}  &\textbf{Nachteile} \\ 
			\hline
			&\\
			\begin{minipage}[t]{0.45\textwidth}
				\begin{itemize}
					\item sehr feines Mahlen möglich
					\item großer Zerkleinerungsgrad \\
					$z=\zeta=\frac{x_{80,\alpha}}{x_{80,\omega}}$
					\item enge Korngrößenverteilung, wegen vorrangiger Zerkleinerung großer Teilchen
					\item Mahlkörper können dem Mahlgut angepasst werden (Material, Größe)
					\item Autogenes Mahlen möglich
					\begin{itemize}
						\item Mahlgut selbst ist Mahlkörper
						\item Mahlkörper werden durch Abrieb immer kleiner \\
						(Abrieb = Produkt)
						\item Mahlkörper müssen immer weiter zugegeben werden
					\end{itemize}
				\end{itemize}
			\end{minipage} & 
			\begin{minipage}[t]{0.35\textwidth}
				\begin{itemize}
					\item sehr energieaufwendig (Kugel zu heben kostet eben)
					\item trennen von Mahlgut und Mahlkörper erforderlich
					\item Lärm
				\end{itemize}
			\end{minipage}\\
		\end{tabulary}}
	\end{table}
	\FloatBarrier
	\vspace*{-2.5mm}
	%Tabelle ENDE
\end{itemize}

\newpage






