\chapter{Trennen}

\section{Stoffgemische}

%Tabelle START

\vspace*{-2.5mm}
\renewcommand{\arraystretch}{1.2}
\begin{table}[h!]
	\centering
	\caption{Stoffgemische}
	\label{tab:tabelle1}
	%\resizebox{10cm}{!}{
	\begin{tabulary}{\textwidth}{C|L}
		\hline
		\textbf{Kombination der Phasen} & \textbf{Bezeichnung}\\
		\hline
		S in G & Rauch, Staub,...\\
		S in L (Aerosol)& Suspension, Schlamm, Trübe,...\\
		L in G (Aerosol)& Dampfwolken, Nebel, Regen, Sprühwolke,...\\
		G in L & Sprudelschicht, Blasenschwarm, Schaum,...\\
		L in L & Emulsion, Tropfenschwarm\\
	\end{tabulary}
	%}
\end{table}
\FloatBarrier
\vspace*{-2.5mm}

%Tabelle Ende

\section{Trennverfahren}

Alle Stoffsysteme sind dispers und bestehen aus mindestens 2 Phasen.\\
Nur dann kann man \underline{mechanische Trennverfahren} anwenden. \\
(Grenze nach unten ist dabei die Partikelgröße)\\ \\
Für einphasige Stoffsysteme müssen thermische Trennverfahren angewendet werden.\\
Mechanische Verfahren sind meist effizienter als thermische Verfahren.

\begin{itemize}
	\item \textbf{Sedimentation} $\approx \SI{10}{\micro \meter}$ \begin{small}(S/G, S/L, L/L, G/L, L/G)\end{small}\\
	= Absetzen/Aufsteigen von Teilchen im Schwerkraftfeld \\ \\
	$\rightarrow$ \textit{Voraussetzung:} \\
	unterschiedliche Dichte der Teilchen gegenüber Fluid
		
	\item \textbf{Zentrifugation} $< \SI{10}{\micro \meter}$ \begin{small}(S/L)\end{small}\\
	= Trennen im Zentrifugalfeld \\ \\
	$\rightarrow$ geeignet für sehr geringe Dichteunterschiede und sehr kleine Teilchen
	
	\item \textbf{Filtration} \begin{small}(S/G, S/L)\end{small}\\
	= Teilchendurchmesser > Porendurchmesser des Filtermediums \\
	"`Sterische (räumliche) Hinderung"' 
	
	\item \textbf{Sieben} \begin{small}(S/G)\end{small}\\
	= Trennen nach Größenunterschied \\
	$\rightarrow$ Klassierung
	
	\item \textbf{Sichten} \begin{small}(S/G)\end{small}\\
	= Trennen nach Luftwiderstand und Dichte
	
	\item \textbf{Flotation} \begin{small}(S/S/G)\end{small}\\
	= spezielle Sedimentation
	
	\item \textbf{Zyklon} \begin{small}(S/G, S/L)\end{small}\\
	= ähnlich wie Zentrifugation
\end{itemize}

\subsection{Sedimentation}
= Absetzen einer dispersen Phase unter Einwirkung der Schwerkraft\\

Disperse Phase kann eine höhere oder niedrigere Dichte haben, als die Kontinuierliche.\\
$\rightarrow$ wichtige Trennoperation, weil Apparate \underline{einfach} und somit \underline{günstig} sind\\

\textbf{Bezeichnung des Sediments nach Zweck}
\begin{itemize}
	\item \textit{Klären:} \\
	Trennziel = klare Flüssigkeit mit möglichst wenig Teilchen
	\item \textit{Eindicken:} \\
	Trennziel = möglichst konzentrierter Schlamm mit möglichst wenig Flüssigkeit
\end{itemize}



\newpage

\section{Grundlagen der Modellierung}
Bewegung eines Einzelteilchens im Schwerkraftfeld\\
$\rightarrow$ Annahme: Teilchen ist starr, kugelförmig und glatt\\ \\
$d_P>\SI{10}{\micro \meter}$\\
$\rho_P>\rho_F$
\vspace*{-20mm}
%Start
\begin{figure}[h!]
	\centering
	\includegraphics[width=0.22\textwidth]{img/partikelmodell}
	\caption{Skizze eines Partikels}
	\label{skizzepruef}
\end{figure}
\FloatBarrier
%Ende
\vspace*{-5mm}
\begin{flalign}
F_G &= m_P*g=V_P*\rho_P*g=\frac{\pi}{6}*d_P^3*\rho_P*g\\
F_T &= m_F*g=V_P*\rho_F*g=\frac{\pi}{6}*d_P^3*\rho_F*g
\end{flalign}
\begin{small}\begin{center}"'Auftrieb ist Masse der verdrängten Flüssigkeit"'\end{center}\end{small}
\vspace*{-5mm}
\begin{flalign}
F_R &= c_W*\rho_F*\frac{1}{2}*v_P^2*A_\perp
\end{flalign}


$c_W$... Widerstandsbeiwert $c_W=f(v,\text{Geometrie}, \text{Rauigkeit},...)$\\
$v_P$... Relativgeschwindigkeit zwischen Teilchen und Partikel\\
$A_\perp$... Projezierte Fläche des Partikels in Bewegungsrichtung\\
hier: Kugel $\rightarrow$ Kreis mit $A_\perp=\frac{\pi}{4}*d_P^2$

\newpage

\section{Grundlagen der Modellierung}
Bewegung eines Einzelteilchens im Schwerkraftfeld\\

\textbf{\underline{Annahme Teilchen ist:}} 
\begin{itemize}
	\item kugelförmig
	\item starr
	\item glatt
	\item $d_P>\SI{10}{\micro \meter}$
	\item $\rho_P>\rho_F$
\end{itemize}

\textbf{\underline{Kräfte die wirken:}}
\begin{itemize}
	\item[a)] $\overrightarrow{F_G}$: Gewichtskraft
	\item[b)] $\overrightarrow{F_T}$: Trägheitskraft {\footnotesize (bei kleineren Partikeln eher uninteressant)}
	\item [c)] $\overrightarrow{F_A}$: Auftriebskraft
	\item [d)] $\overrightarrow{F_R}$: Reibungskraft
\end{itemize}
\newpage
\textbf{\underline{Kräftegleichgewicht:}}
\begin{flalign}
	0 &= \overrightarrow{F_G}+\overrightarrow{F_T}+\overrightarrow{F_A}+\overrightarrow{F_R}
\end{flalign}
\begin{itemize}
	\item[a)] $\overrightarrow{F_G} = m*g=V_P*\rho_P*g=\frac{\pi}{6}*(d_P)^3*\rho_P*g $
	\item[b)] $\overrightarrow{F_T} = V_P*\left(\rho_P+c_W*\rho_F\right)\frac{\text{d}v_P}{\text{d}t}$
	\begin{description}
		\item[$\boldsymbol{m_P}$...] $=V_P*\rho_P$
		\item[$\boldsymbol{m_F}$...] $=V_P*c_m*\rho_F$ {\footnotesize (für Kugeln $c_m=0,5$)}
	\end{description}
	$m_F$ ist Masse des umgebenden Fluids das mit Partikel mitgerissen und beschleunigt wird (\textit{Schleppwirbel})
	\item [c)] {\footnotesize "`\textit{Auftrieb ist Masse der verdrängten Flüssigkeit}"'}\\
	$\overrightarrow{F_A} = m_F*g=V_P*\rho_F*g=\frac{\pi}{6}*(d_P)^3*\rho_F*g $
	\item [d)] $\overrightarrow{F_R}=c_W*\rho_F*\frac{1}{2}*(v_P)^2*A_\bot $
	\begin{description}
		\item[$\boldsymbol{c_W}...$] Widerstandsbeiwert $c_W =f(V,Geometrie, Rauhigkeit,...)$
		\item[$\boldsymbol{v_P}...$] Relativgeschwindigkeit zwischen Partikel und Fluid
		\item[$\boldsymbol{A_\bot}...$] projezierte Fläche des Partikels in Bewegungsrichtung\\
		hier: Kugel $\rightarrow$ Kreis $\o d_P$ $\rightarrow A_\bot=\frac{\pi}{4}*(d_P)^2$
	\end{description}
\end{itemize}

ABBILDUNG\\

\subsubsection{Differentialgleichung der Bewegung einer starren Kugel im Schwerkraftfeld:}
\begin{flalign}
	\frac{\text{d}v_P}{\text{d}t} &= \frac{g*|\rho_P-\rho_F|}{\rho_P+\frac{\rho_F}{2}}-\frac{3*c_W*\rho_F*(v_P)^2}{4*d_P*\left(\rho_P+\frac{\rho_F}{2}\right)}
\end{flalign}

Bei der Sedimentation von kleinen Teilchen wird davon ausgegangen, dass die Beschleunigungsphase sehr kurz ist. \\
$\rightarrow$ \textit{kann deshalb vernachlässigt werden}
\begin{flalign}
	\frac{\text{d}v_P}{\text{d}t}&=0
\end{flalign}
Teilchen haben eine konstante Geschwindigkeit \\
$\rightarrow$ \textit{stationäre Sedimentation(sgeschwindigkeit)}
\newpage

\subsubsection{Umschlagpunkt von laminar $\rightarrow$ Übergang/turbulent}
\begin{itemize}
	\item in Rohrleitungen bei $Re_R=2300$
	\item Umströmung von Partikeln bei $Re_P=1$
\end{itemize}

ABBILDUNG\\

\begin{itemize}
	\item[\textbf{a)}] \underline{\textbf{laminarer Bereich}} \quad \quad $Re_P<1 \quad (\text{Zogg:} Re<0,2)$\\
	In diesem Bereich ist $c_W$-Wert genau definiert
	\begin{flalign}
		c_W	&= \frac{24}{Re_P}
	\end{flalign}
	\item[\textbf{b)}] \underline{\textbf{Übergangsbereich}} \quad \quad $1<Re_P<10^4 $\\
	In diesem Bereich existieren viele Näherungsgleichungen 
	\begin{flalign}
	c_W	&= \frac{1}{3}\left(\sqrt{\frac{72}{Re_P}}+1\right)^2
	\end{flalign}
	\item[\textbf{c)}] \underline{\textbf{turbulenter Bereich}} \quad \quad $Re_P>10^4 $
	\begin{flalign}
	c_W	&= 0,44 = const.
	\end{flalign}
\end{itemize}

\section[Archimediszahl $Ar$]{Archimediszahl $\boldsymbol{Ar [-]}$}
\begin{flalign}
	Ar &= \frac{g*(d_P)^3*|\rho_p-\rho_F|*\rho_F}{(\eta_F)^2}
\end{flalign}
\textbf{\underline{Ziel:}} Berechnung der Sinkgeschwindigkeit\\
\begin{itemize}
	\item \textbf{laminare Strömung} \quad $\boldsymbol{Re<0,2}$ \textbf{ und } $\boldsymbol{Ar<3,6}$
	\begin{flalign}
		Re_P &= \frac{Ar}{18}
	\end{flalign}
	\begin{flalign}
	v_P &= \frac{|\rho_P-\rho_F|*g*(d_P)^2}{18*\eta_F}
	\end{flalign}
	\item \textbf{Übergangsbereich} \quad $\boldsymbol{0,2<Re<10^4}$ \textbf{ und } $\boldsymbol{3,6<Ar<10^{10}}$\\
	es existieren viele Näherungsgleichungen für $Re_P$
	\begin{flalign}
	Re_P &= 18*\left(\sqrt{1+\frac{\sqrt{Ar}}{9}}-1\right)^2
	\end{flalign}
	\begin{flalign}
	v_P &= Re_P*\frac{\eta_F}{\rho_F*d_P}
	\end{flalign}
	\item \textbf{turbulente Strömung}\\
	$\rightarrow$ ist für die Sedimentation nicht relevant
\end{itemize}

\vspace*{10mm}

\textsc{{\Large Vorangegangene Gleichungen gelten nur für Einzelteilchen!!}}\\ \\

%\newpage

Bei der Sedimentation im \underline{Teilchenschwarm} wird de Sinkgeschwindigkeit gering, weil:
\begin{itemize}
	\item die Teilchen behindern sich gegenseitig (besonders bei unterschiedlich großen Teilchen)
	\item Jedes Teilchen transportiert Flüssigkeit im Schleppenwirbel mit nach unten\\
	$\rightarrow$ \textbf{Folge:} Es entsteht eine Aufwärtsströmung\\
	Deswegen muss die Sinkgeschwindigkeit, die berechnet wurde, für das Einzelteilchen angepasst werden $\rightarrow$ siehe Diagramm \textsc{Zogg}
\end{itemize}

ABBILDUNG \\

\textbf{"'Teilchenvolumenanteil"' $\boldsymbol{\kappa}$:}
\begin{flalign}
	\chi	&= \frac{m_P \text{{\tiny  (trocken)}}}{m_F}\\[1mm]
	\kappa 	&= \frac{\chi}{\chi+\frac{\rho_P}{\rho_F}}
\end{flalign}

\newpage

\section{Auslegung von Schwerkraftsedimentationsapparaten}

ABBILDUNG\\

\underline{\textbf{Prozessziel:}}
\begin{itemize}
	\item Klären: \quad $\rho_{s,\omega_1} \approx 0$
	\item Eindicker: \quad $\rho_{s,\omega_2} \approx 0,4...0,5$ mineralische Stoffe\\
	\hspace*{2.7cm} $\rho_{s,\omega_2} \approx 0,65...9,0$ biologische Stoffe
\end{itemize}

ABBILDUNG\\

Die Absetzvorgänge treten gleichermaßen auch im durchströmten Sedimentationsapparat auf. Das Material aus der Kompressionszone soll nur ausgetragen werden, wenn es ausreichend konzentriert wird.

\section{Auslegung eines Klärbeckens}

ABBILDUNG\\

\begin{itemize}
	\item \textbf{gegeben:} \quad $\dot{V}_{\alpha}, \rho_{\alpha}$
	\item \textbf{gefordert:} \quad $\rho_{\omega_1}, \rho_{\omega_2}$
	\item \textbf{gesucht:} \quad $\dot{V}_{\omega_1},\dot{V}_{\omega_2}, l, s, b$
\end{itemize}

\subsubsection{Annahmen:}
\begin{itemize}
	\item Zulauf ist eine ideale, beruhigte, horizontale Propfenströmung (gleichmäßig über den Behälterquerschnitt) $A_\bot$ verteilt:\\
	Horizontalgeschwindigkeit $v_B = const.$
	\item Einlaufzone wird nicht zu $l$ gezählt
	\item Horizontalgeschwindigkeit $v_B$ wird nur durch $\dot{V}_{\omega_1}$ bestimmt (d.h. der Feststoffanteil wird vernachlässigt)
	\item trotz des Absetzens von Feststoffen wird die vertikale Komponente der klaren Phase vernachlässigt (ist in $v_P$ integriert)
\end{itemize}

\newpage

\subsubsection{Herleitung:}
$\rightarrow$ \textit{Zeit für das Absinken eines Teilchens von der Oberfläche bis zum Boden:}
\begin{flalign}
	t	&= \frac{s \left[\si{\meter}\right]}{v_P \left[\si{\meter \per \second}\right]}
\end{flalign}
$\rightarrow$ \textit{Zeit für die horizontale Durchströmung des Behälters mit:}
\begin{flalign}
	v_B	&= \frac{\dot{V}_{\alpha}}{A_{\bot}}\approx \frac{\dot{V}_{\omega_1}}{A_{\bot}} =\frac{\dot{V}_{\omega_1}}{s*b}
\end{flalign}
\begin{flalign}
	t	&= \frac{l \left[\si{\meter}\right]}{v_B \left[\si{\meter \per \second}\right]}	
\end{flalign}
$\rightarrow$ \textit{Gleichsetzen ergibt:}
\begin{flalign}
	\frac{\cancel{s}}{v_P}	&= \frac{l}{v_B} = \frac{l*\cancel{s}*b}{\dot{V}_{\omega_1}}
\end{flalign}
$\rightarrow$ \textit{Länge des Beckens $l$:}
\begin{flalign}
	l	&= \frac{\dot{V}_{\omega_1}}{v_P*b}\\[1mm]
	v_P	&= \frac{\dot{V}_{\omega_1}}{l*b} =\frac{\dot{V}_{\omega_1}}{A_o}
\end{flalign}
\begin{description}
	\item[$\boldsymbol{A_o}$...] Oberfläche des Beckens "`Klärfläche"'
\end{description}
\vspace*{5mm}
\textsc{{\Large Gleichung ist \underline{unabhängig} von Tiefe des Beckens}}\\ \\

\newpage

\textsc{{\Large \underline{Aber:}}}\\
$v_B$ darf nur so groß sein, dass die Teilchen nicht aufgewirbelt werden, d.h. im Becken muss eine laminare Strömung vorliegen, d.h.:
\begin{flalign}
	Re &<2000 \quad {\footnotesize \text{(max. 2300 laminar)}}
\end{flalign}
\textbf{\underline{weitere Auslegungsbedingung:}}\\
\begin{flalign}
	Re_B	&= \frac{v_B*d_{hydr}*\rho_F}{\eta_F}
\end{flalign}
\begin{description}
	\item[$\boldsymbol{d_{hydr}}$...] hydraulischer Durchmesser\\
	\begin{flalign}
		d_{hydr}	&= \frac{4*A_{\bot}}{\text{benetzter Umfang}}
	\end{flalign}
\end{description}

\underline{\textbf{Beispiel:}}\\
\begin{itemize}
	\item quadratischer Kanal \hspace*{1.8cm} $d_{hydr} =\frac{\cancel{4}*a^{\cancel{2}}}{\cancel{4}*\cancel{a}}=a$
	\item offenes, rechteckiges Becken \quad $d_{hydr}=\frac{4*(s*b)}{2s+b}$
\end{itemize}

\textbf{damit ist:}\\
\begin{flalign}
	Re_B 	&= \frac{4*s*b*v_B*\rho}{(2s+b)*\eta_F}\\[1mm]
			&= \frac{4*\dot{V}_{\omega_1}*\rho_F}{(b+2s)*\eta_F}
\end{flalign}
\textbf{weiteres Kriterium (Erfahrungswert):} \\
\begin{flalign}
		\frac{b}{s} = 2...4 \quad \text{\footnotesize (rechteckiges Klärbecken)}
\end{flalign}

ABBILDUNG\\

\textit{Beckentiefe kommt in der Auslegungsgleichung nicht vor !}
\begin{flalign}
	v_P	&= \frac{\dot{V}_{\omega_1}}{A_o}
\end{flalign}
\textbf{\underline{Aber:}} laminare Durchströmung ! $\rightarrow$ dadurch ergibt sich die Beckentiefe

\subsubsection{Unterteilung des Beckens in Stapeln der Absetzzonen}

ABBILDUNG\\

Die gesamte Klärfläche $A_o$ wird auf eine viel kleine Grundfläche $A_G$ gebracht.\\

ABBILDUNG\\

Schrägstellen des Plattenpakets hilft beim Schlammaustrag \linebreak
(Abrutschen des Schlamms) $\rightarrow$ \textbf{Lamellenklärer/Schrägklärer}

\section{Auslegung von G/L und L/L Abscheidern}

Die Behälter können horizontal als auch vertikal sein.
\begin{itemize}
	\item L/L $\rightarrow$ immer horizontal
	\item G/L 
		\begin{itemize}
			\item [$\rightarrow$] hoher Flüssigkeitsanteil $\geq 10\%$: horizontal
			\item [$\rightarrow$] geringer Flüssigkeitsanteil $\leq 10\%$: vertikal
		\end{itemize}
\end{itemize}

Die Auslegung von horizontalen Abscheidern ist äquivalent zum Klärbecken.\\

ABBILDUNG\\

Der Behälter ist zur Hälfte mit Flüssigkeit (bzw. schwere flüssige Phase) gefüllt. \\
Die (vertikale) Absetzlänge ist der halbe Behälterdurchmesser.\\

\textbf{\textit{Koaleszens:}}\\
Vereinigung von Tropfen mit kontinuierlicher Flüssigkeit oder das Zusammenfließen von 2 Tropfen.\\

ABBILDUNG\\

\subsubsection{Vertikaler Absetzbehälter}

ABBILDUNG \\

\textbf{Schwebezustand eines Tropfens:}
\begin{flalign}
	|v_P|	&= |v_B|
\end{flalign}
\textit{Dies ist auch der Grenzfall für die Auslegung}\\

\textbf{Absetzbedingung:}
\begin{flalign}
|v_P|	&> |v_B|
\end{flalign}

\textbf{Erfahrungswerte:}
\begin{itemize}
	\item $d_P= \SI{0,5}{\milli \meter}=\SI{500}{\micro \meter}$
	\item Trennhöhe $H=D$
	\item $H_{\text{Zulauf}}=\frac{1}{4} D$
	\item $H_L$ aus der $L-$Verweilzeit$=3...10$min
	\item $H+H_{\text{zul}}+H_L\leq L$
	\item $\frac{L}{D}=3$ (für niedrige Drücke unter \SI{20}{\bar})$\rightarrow$ wirtschaftlich + Behälterbau
	\item $Re_P \approx$ meist Übergangsbereich
\end{itemize}

\textit{Behälter} = gunstig in der Investition, teuer im Betrieb\\
\textit{Zentrifuge} = teuer in der Investition, günstig im Betrieb

\newpage

\section{Flockung, Flokkuation}
Für ein Teilchen kleiner \SI{10}{\micro \meter} ergeben sich sehr lange Absetzzeiten.\linebreak
Teilchen kleiner \SI{1}{\micro \meter} setzen sich unter Schwerkraftwirkung gar nicht mehr ab (\textsc{Brown}'sche Molekülbewegung).\\
Durch Zugabe von Flockungsmitteln können die kleinen Teilchen agglomerieren, sie bilden größere Verbände (Flocken). Die (größeren) Flocken setzen sich schneller ab. Flocken sind sehr instabil. \linebreak
{\small (Vorsicht bei Pumpen oder anderen hohen Strömungsgeschwindigkeiten!)}\\ \\
$\rightarrow$ Auslegung der Apparate wie oben \\ \\
Die Auswahl des geeigneten Flockungsmittels und die Flockungseignung von Teilchen und die sich daraus ergebenen Absetzzeiten sind empirisch zu bestimmen.\\ \\
\textbf{\underline{Aber:}}\\
Flockungsmittel sind teuer (Betriebskosten) und dürfen keine "`Nebenwirkungen"' haben (Toxikologie, Umwelt, Entsorgung)

\section{Flotation}
Schwerkrafttrennung unter Zuhilfenahme von Luft. \linebreak
Es können zwei verschiedene Feststoffe voneinander getrennt werden.\\

ABBILDUNG 

\begin{itemize}
	\item \textbf{Prinzip:}\\
	ein Feststoff lagert sich an Luftblasen an, der andere nicht.\linebreak
	eventuell muss ein Flotationhilfsmittel (Tenside) hinzugegeben werden, um die hydrophobe Eigenschaft einzustellen\\
	
	ABBILDUNG\\
	
	\item \textbf{Anwendung:}\\
	Papierrecycling, Erzaufbereitung
\end{itemize}


\section{Zentrifugation}
= Sedimentation im Zentrifugalkraftfeld
\begin{itemize}
	\item [$\rightarrow$] bei sehr kleinen Teilchen ($d_P<\SI{10}{\micro \meter}$) oder bei sehr geringen Dichteunterschieden würde die Sedimentation im Schwerkraftfeld sehr lange dauern bzw. unmöglich sein.
	\item [$\rightarrow$] \textbf{Dann}: Ersatz der Erdbeschleunigung in Form der Zentrifugalbeschleunigung
	\item [$\rightarrow$] \textbf{Apparat}: Zentrifuge\\
		\underline{Aber:} Zentrifugen sind sehr aufwendige und damit teure Apparate (Invest- + und Betriebskosten)
	\item [$\rightarrow$] Charakterisierung von Zentrifugen nach Angabe von x-Mal Erdbeschleunigung (z.B. $100g$)
\end{itemize}

\begin{itemize}
	\item \textbf{Sedimentationszentrifuge:}
			\begin{itemize}
				\item 	ABBILDUNG
				\item Flüssigkeit läuft über (kontinuierlich)
				\item Feststoff setzt sich ab (Sediment)
				\item Austrag des FS (meist diskontinuierlich)
			\end{itemize}
	\item \textbf{Filter- bzw. Siebzentrifuge:}
		\begin{itemize}
			\item ABBILDUNG
			\item Trommel ist perforiert (Sieb) und eventuell mit einem Filtermedium
			\item Flüssigkeit läuft durch bereits Sedimentiertes Material
			\item Feststoffaustrag ist diskontinuierlich oder kontinuierlich (Schubzentrifuge)
		\end{itemize}
\end{itemize}

\begin{itemize}
	\item [$\rightarrow$] \textbf{weitere Bauarten:} Dekanter, Tellerzentrifuge (Siehe Arbeitsblatt)
\end{itemize}
